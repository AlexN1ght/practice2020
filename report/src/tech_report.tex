\begin{center}
\bfseries{\large ТЕХНИЧЕСКИЙ ОТЧЁТ ПО ПРАКТИКЕ}
\end{center}

\section*{Архитектура}
face\_rec.py - В файле представленны функции по распознаванию лиц и для работы с базами данных. В базе данных хранятся пары: кодировка лица/имя. Используются вспомогательные функции из модулей cv2 и face\_recognition, такие как кодирование лица по характерным чертам(face\_encodings), нахождение расположения лиц(face\_locations), сравнение(compare\_faces) лиц.\\
tg\_bot.py - реализация телеграмм бота. Бот телеграмм по сути своей является лишь интерфейсом для работы с нашим сервисом. Сам бот находится на серверах телеграмма и только принимает запросы от пользователя и передает их ему. Но обрабатывать все эти запросы уже должен наш сервер, где и будет находиться вся логика бота. \\
BD.sm и PrivateBD.sm - публичная и приватные базы данных, записаны в бинарный файл.

\section*{Описание}
Это телеграм бот, предназначенный для распознования лиц, добавленных в базу данных. Каждый человек может добавить себя или знакомого и распознать всех людей на фотографии, которые были добавлены в базу. После поиска можно получить фотографию со всеми найденными лицами, а также список имен людей, найденных на фотографии. 

\section*{Реализация}

Регистрация бота и получение токена происходит через другого телеграмм бота @BotFather. Чтобы отправить сообщение на сервер нужно сделать запрос, который оправляется по протоколу HTTP на сервера телеграмм с уникальным идентификатором нашего бота,   который конфиденциален, так как с помощью его можно управлять нашим ботом. Ответ придёт в виде  JSON-объекта,  в котором всегда будет булево поле  ok  и опциональное строковое поле  description,  содержащее человекочитаемое описание результата.\\
Все запросы выполняются с помощью функций, описанных в face\_rec.py. При загрузке нового лица в базу данных мы используем метод load\_image\_file, который позволяет подгрузить изображение и делает его удобным для дальнейшей работы. Для храния лица в базе данных, а также сравнения лиц, мы используем метод face\_encodings, который кодирует лицо по его характерным чертам, это представляет собой список из +-50 флотов. Так операция сравнения лиц переходит к сранению этих вещественных чисел. Для нахождения всех лиц на фото используем метод face\_locations, основанный на cnn(Convolutional Neural Network), который определяет локации лиц на фото, далее можно передать этот параметр в face\_encodings и закодировать все лица на фото, что позволяет сравнивать их с лицами из базы данных. И в завершинии с помощью методов cv2 мы обводим лица прямоугольниками, используя их локации.    

\section*{Тестирование}

Для тестирования в базу данных добавлялись презеденты стран, затем для распознования загружались фотографии с разных саммитов и тд. Также протестированы все основные функции при работе с нашими личными фотографиями.

\section*{Ссылка на GitHub} https://github.com/AlexN1ght/practice2020
\pagebreak
