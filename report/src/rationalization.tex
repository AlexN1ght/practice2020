\begin{center}
\bfseries{\large МАТЕРИАЛЫ ПО РАЦИОНАЛИЗАТОРСКИМ ПРЕДЛОЖЕНИЯМ}
\end{center}
Телеграм бот работает корректно с несколькими пользователями одновременно. Базы данных хранят лишь кодировку лиц, а не сами лица, что более безопасно. Однако в реальных условиях фотографии имеют разные качества, освещения и тд., что может приводить к ошибкам в распознавании лиц. Качество рапознавания можно увеличить, используя более мощные средства распознавания. Наш способ по улучшению распознавателя заключается в следующем: добавлям в бд под тем же именем каждое лицо, которое было распознано и нашлось совпадение. Недостаток очевиден: при неправильном определнии лица дальнейший шанс на правильное распознание уменьшается. Хотя и есть возможность редактировать базы данных, это можно делать исключительно вручную. Если автоматизировать этот процесс, то точность возрастет многократно.  
\pagebreak
